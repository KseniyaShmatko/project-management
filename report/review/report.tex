\documentclass[14pt]{extarticle}
\usepackage[T1]{fontenc}
\usepackage[utf8]{inputenc}
\usepackage{amsmath,amssymb}
\usepackage[russian]{babel}
\usepackage{geometry}
\geometry{a4paper,
total={170mm,257mm},left=2cm,right=1cm,
top=1.5cm,bottom=1.5cm}

\usepackage{tabularx}
\usepackage{tikz}

\newcommand{\podpis}[2]{
    \parbox[b]{8cm}{#1}
    \hspace{1cm}
    \tikz[baseline=2pt]{\draw(0,0) to node[below=-2pt]{\scriptsize подпись}(3cm,0);}
    \hspace{1cm}
    \tikz[baseline=2pt]{
        \def\familywidth{\textwidth-8cm-1cm-3cm-1cm-20pt}
        \draw(0,0) to node[below=-2pt]{\scriptsize дата}(\familywidth,0);
        \node[anchor=west](f) at (25pt,8pt){#2};
    }
}

\begin{document}

\pagestyle{empty}

\begin{center}
\section*{РЕЦЕНЗИЯ}
на выпускную квалификационную работу бакалавра

Шматко Ксении Максимовны

<<Программно-алгоритмический комплекс с многоцелевой и масштабируемой архитектурой для совместной работы и управления проектами>>
\end{center}	

Работа посвящена разработке современного программного комплекса, предназначенного для организации совместной деятельности и управления проектами, что актуально в условиях цифровизации и востребованности гибридных и удалённых форм организации работы.

В аналитическом разделе выпускной квалификационной работы проведён системный обзор современных баз данных, рассмотрены основные модели хранения, проанализированы их сильные и слабые стороны. Особое внимание уделено возможностям работы с неоднородными и структурированными данными.

В конструкторском разделе подробно описана архитектура комплекса, охватывающая все ключевые компоненты. Подчёркиваются модульность, расширяемость и чёткость взаимодействия между компонентами системы.

В технологическом разделе произведен выбор средств реализации, показано разработанное программное обеспечение и методы его тестирования. 

В исследовательском разделе рассмотрены сценарии полноценной совместной работы над проектами, характеристики комплекса. Система построена так, что легко расширяется новыми типами данных и сервисов.

К достоинствам разработанного ПО можно отнести архитектуру с раздельным хранением структурированных и неструктурированных данных, модульность и спроектированные взаимодействия между слоями, гибридность модели хранения и организацию моделей, позволяющих управлять разнообразным контентом.

К недостаткам можно отнести ограниченность набора реализованных сервисов и отсутствие реализации в полном объёме обмена изменениями между пользователями в реальном времени.

Несмотря на отмеченные недостатки, считаю, что выпускная квалификационная работа Шматко~К.~М. <<Программно-алгоритмический комплекс с многоцелевой и масштабируемой архитектурой для совместной работы и управления проектами>> соответствует квалификационным требованиям, предъявляемым к выпускной квалификационной работе бакалавра, заслуживает отличной оценки, а Шматко~К.~М.~---~присвоения квалификации бакалавра по направлению подготовки 09.03.04 <<Программная инженерия>>.

\vspace{1.5cm}

\setlength{\parindent}{0pt}

\podpis{Рецензент:\\Руководитель команды разработки бэкенда инструментов привлечения\\ООО <<Яндекс Фантех>>\\Филютин Алексей Александрович}{29.05.2025}\\

\end{document}