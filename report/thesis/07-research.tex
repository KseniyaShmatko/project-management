\chapter{Исследовательский раздел}

\section{Исследование применимости разработанного программного обеспечения}

Целью разработки программно-алгоритмического комплекса являлось создание системы для совместной работы над проектами с возможностью управления неоднородными данными. 
Исследование применимости основывается на сопоставлении реализованного функционала с исходными задачами, сформулированными ранее.

\subsection{Соответствие основным задачам}

На текущем этапе разработки комплекс успешно решает следующие ключевые задачи:

\begin{enumerate}
    \item \textbf{Регистрация и аутентификация пользователей:} Система предоставляет функционал для создания учетных записей и безопасного входа пользователей с использованием JWT, что является базовым требованием для многопользовательских систем.
    \item \textbf{Создание и управление проектами:} Пользователи могут создавать проекты, являющиеся контейнерами для дальнейшей работы. Реализовано присвоение роли владельца создателю проекта.
    \item \textbf{Управление доступом к проектам:} Владелец проекта может добавлять других зарегистрированных пользователей в проект и назначать им роли (<<Редактор>>, <<Читатель>>), а также удалять участников. Это обеспечивает разграничение прав доступа к информации внутри проекта.
    \item \textbf{Работа с файлами типа "заметка":} Реализована возможность создания, просмотра, переименования и редактирования текстовых документов (заметок) внутри проектов.
    \item \textbf{Редактирование контента:} Редактор позволяет пользователям создавать структурированный контент, используя различные блоки (текст, заголовки, списки, изображения). Реализована загрузка изображений на сервер и их отображение в редакторе. Данные редактора сохраняются в MongoDB, что соответствует требованию хранения неоднородного контента.
\end{enumerate}
Данный набор реализованных функций подтверждает применимость разработанного ПО для организации базовой совместной работы с текстовыми документами в рамках проектной деятельности. 
Пользователи могут создавать изолированные рабочие пространства (проекты), управлять их содержимым (заметками) и контролировать доступ других участников.

\subsection{Потенциал для реализации других сервисов}

Заложенная архитектура и модель данных (в частности, концепция SuperObject и ContentBlock в MongoDB, а также таблица file\_types в PostgreSQL) создают основу для последующей интеграции остальных сервисов (схем, презентаций, трекера задач и т.д.):
\begin{itemize}
    \item \textbf{Поддержка различных типов файлов:} Система SuperObject с полем serviceType позволяет добавлять новые типы контента. Для каждого нового типа достаточно будет разработать соответствующий клиентский редактор и, при необходимости, специфическую логику обработки на сервере. Метаданные нового типа файла могут быть добавлены в таблицу file\_types.
    \item \textbf{Гибкость хранения контента:} Использование MongoDB для SuperObject и связанных с ним блоков (или специфичных для сервиса коллекций) обеспечивает необходимую гибкость для хранения данных различной структуры, что важно для таких сервисов, как трекер задач или календарь.
\end{itemize}
Таким образом, применимость комплекса распространяется и на нереализованные на данный момент задачи, благодаря заложенной расширяемой архитектуре.

\clearpage
\section{Исследование характеристик разработанного комплекса}

Исследование характеристик комплекса проводится на основе анализа выбранных технологий, архитектурных решений и реализованного функционала.

\subsection{Гибкость и расширяемость}

\begin{itemize}
    \item \textbf{Модель данных для неоднородного контента:} Принятый подход с использованием SuperObject в MongoDB как центральной сущности для представления файла (независимо от его внутреннего типа) и связанных с ним блоков ContentBlock (для блочных типов контента типа заметок или презентаций) или специфичных коллекций (для схем, календарей и т.д.) обеспечивает высокую гибкость. Система может быть расширена для поддержки новых типов документов и сервисов без значительного изменения основной архитектуры хранения. Добавление нового типа контента сведется к:
        \begin{enumerate}
            \item Определению нового значения для serviceType в SuperObject.
            \item Добавлению записи в таблицу file\_types (PostgreSQL).
            \item Реализации соответствующего редактора на клиентской стороне.
            \item При необходимости, разработке специфических коллекций в MongoDB и API-эндпоинтов для обработки данного типа контента.
        \end{enumerate}
    \item \textbf{Модульность серверного приложения:} Использование Spring Boot и разделение логики на контроллеры, сервисы и репозитории способствует модульности. Новые функциональные блоки могут быть добавлены как отдельные сервисы, минимизируя влияние на существующий код.
    \item \textbf{Компонентная архитектура на клиенте:} Применение React позволяет создавать переиспользуемые UI-компоненты, что упрощает добавление новых интерфейсов и модификацию существующих.
\end{itemize}
Эти факторы указывают на способность системы к расширению и адаптации под новые требования.

\subsection{Масштабируемость}

Полноценное нагрузочное тестирование не проводилось, однако архитектурные решения позволяют сделать качественную оценку потенциала масштабируемости:
\begin{itemize}
    \item \textbf{Горизонтальная масштабируемость MongoDB:} MongoDB изначально спроектирована для горизонтального масштабирования (шардинг, репликация), что позволяет распределять нагрузку при хранении и обработке больших объемов контента документов.
    \item \textbf{Масштабируемость PostgreSQL:} Несмотря на то, что PostgreSQL традиционно масштабируется вертикально, существуют решения для горизонтального масштабирования, которые могут быть применены при необходимости. Метаданные обычно занимают меньше места и генерируют меньше нагрузки по сравнению с самим контентом, поэтому встроенных возможностей PostgreSQL может быть достаточно на длительный период.
    \item \textbf{Stateless Backend:} Серверное приложение на Spring Boot, использующее JWT для аутентификации, спроектировано как stateless. Это означает, что каждый запрос содержит всю необходимую информацию для его обработки, и состояние сессии пользователя не хранится на конкретном экземпляре сервера. Такой подход является ключевым для горизонтального масштабирования серверной части путем запуска нескольких экземпляров приложения за балансировщиком нагрузки.
\end{itemize}
Текущая реализация и выбранный стек технологий создают предпосылки для масштабирования системы при росте числа пользователей и объемов данных, однако для подтверждения потребуются специализированные нагрузочные тесты.

\clearpage
\subsection{Производительность}

Оценка производительности также носит качественный характер:
\begin{itemize}
    \item \textbf{Отклик интерфейса:} Использование React с виртуальным DOM способствует созданию отзывчивого пользовательского интерфейса. Асинхронные запросы к API выполняются без блокировки основного потока. Загрузка данных для проектов и файлов происходит по мере необходимости.
    \item \textbf{Скорость работы с базами данных:} PostgreSQL обеспечивает быструю обработку запросов к структурированным метаданным, особенно при наличии корректных индексов. MongoDB оптимизирована для операций чтения и записи документов, что важно для скорости доступа к содержимому заметок.
    \item \textbf{Сохранение в редакторе:} Реализовано отложенное (debounced) сохранение изменений, что снижает частоту обращений к серверу и базам данных при активном редактировании, не перегружая систему частыми мелкими операциями.
    \item \textbf{Загрузка изображений:} Вынесена в отдельный эндпоинт, обработка файла происходит асинхронно.
\end{itemize}
В ходе ручного тестирования на тестовых данных заметных проблем с производительностью при выполнении основных операций (открытие проектов, файлов, редактирование текста) выявлено не было. 
Однако, для больших объемов данных в одной заметке или при очень большом количестве одновременных пользователей могут потребоваться дополнительные оптимизации (например, пагинация блоков контента, оптимизация запросов к БД).

\section*{Вывод}

Исследование применимости показало, что разработанный программно-алгоритмический комплекс, на текущем этапе его развития, соответствует основным заявленным целям по организации совместной работы с проектами и документами типа "заметка". 
Заложенные архитектурные решения, в частности, модель данных для неоднородного контента и модульная структура, создают прочную основу для дальнейшего расширения функциональности и добавления новых типов сервисов.

Качественный анализ характеристик комплекса выявил хороший потенциал в области гибкости, расширяемости и масштабируемости. 
Выбранный технологический стек (Kotlin/Spring Boot, TypeScript/React, PostgreSQL/MongoDB) позволяет эффективно решать поставленные задачи. 
Хотя количественные метрики производительности и масштабируемости требуют проведения специализированных тестов, текущая реализация демонстрирует приемлемую отзывчивость для основных пользовательских сценариев. 
Удобство использования обеспечивается применением современных UI-компонентов и знакомых паттернов взаимодействия.

Для дальнейшего развития рекомендуется провести нагрузочное тестирование, расширить покрытие автоматизированными тестами и собрать обратную связь от реальных пользователей для улучшения юзабилити и выявления приоритетных направлений доработки.
