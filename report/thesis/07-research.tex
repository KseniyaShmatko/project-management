\chapter{Исследовательский раздел}

\section{Исследование применимости разработанного программного обеспечения}

В условиях исчерпания естественных входных данных для формирования датасетов, необходимых для обучения систем искусственного интеллекта, разработанный программно-алгоритмический комплекс представляет особую ценность.
При его внедрении в образовательный процесс возможно накопление значительных объемов структурированных данных для научных исследований.

На основе учебного плана кафедры ИУ7 <<Программное обеспечение ЭВМ и информационные технологии>> и актуальных данных о контингенте студентов на 2025 год проведена оценка потенциала формирования датасетов при внедрении системы в образовательный процесс~\cite{plan, students}.

\subsubsection{Исходные данные для расчета}

\begin{itemize}[wide=12.5mm, leftmargin=12.5mm]
    \item \textbf{Фактический контингент студентов:}
    \begin{itemize}[wide=12.5mm, leftmargin=12.5mm]
        \item 1 курс — 160 человек.
        \item 2 курс — 117 человек.
        \item 3 курс — 103 человек.
        \item 4 курс — 93 человек.
        \item \textbf{Всего:} 473 человек.
    \end{itemize}
    \item \textbf{Коэффициент активных студентов:} 0,82 (учитывая отчисления и академические отпуски).
    \item \textbf{Фактический контингент активных студентов:} 389 человек.
    \item \textbf{Характер учебных активностей:}
    \begin{itemize}[wide=12.5mm, leftmargin=12.5mm]
        \item По курсовым и НИР: отчеты со схемами.
        \item На гуманитарных дисциплинах: минимум 1 презентация и 2 отчета на студента.
        \item Технические дисциплины: около 60\% преподавателей используют презентации.
    \end{itemize}
\end{itemize}

Приведенные в таблице~\ref{tab:documents} расчеты представляют приближенную оценку документов, которые могут быть созданы в системе.

\begin{table}[h]
    \centering
    \small
    \caption{Потенциальный объем документов при внедрении системы}
    \label{tab:documents}
    \begin{tabular}{|p{5.2cm}|p{7.8cm}|r|}
    \hline
    \textbf{Тип документа} & \textbf{Формула расчёта} & \textbf{Итого} \\
    \hline
    Презентации студентов по гуманитарным дисциплинам & 
    1 курс: 2 дисц. × 160 студ. × 0{,}82 × 1 = 262 \newline
    2 курс: 2 дисц. × 117 студ. × 0{,}82 × 1 = 191 \newline
    3 курс: 2 дисц. × 103 студ. × 0{,}82 × 1 = 168 \newline
    4 курс: 1 дисц. × 93 студ. × 0{,}82 × 1 = 76
    & 697 \\
    \hline
    Отчёты по гуманитарным дисциплинам & 
    1 курс: 2 дисц. × 160 студ. × 0{,}82 × 2 = 524 \newline
    2 курс: 2 дисц. × 117 студ. × 0{,}82 × 2 = 383 \newline
    3 курс: 2 дисц. × 103 студ. × 0{,}82 × 2 = 337 \newline
    4 курс: 1 дисц. × 93 студ. × 0{,}82 × 2 = 152
    & 1,396 \\
    \hline
    Отчёты по курсовым работам и НИР & 
    3 курс: 2 КР × 103 студ. × 0{,}82 = 168 \newline
    4 курс: 2 КР × 93 студ. × 0{,}82 = 152 \newline
    НИР: (103+93) × 0{,}82 = 160
    & 480 \\
    \hline
    Презентации преподавателей & 
    25 дисциплин × 2 преп. × 17 недель × 0{,}6 
    & 510 \\
    \hline
    Отчёты по лабораторным работам & 
    1 курс: 5 дисц. × 160 студ. × 0{,}82 × 6 = 3,936 \newline
    2 курс: 5 дисц. × 117 студ. × 0{,}82 × 6 = 2,878 \newline
    3 курс: 5 дисц. × 103 студ. × 0{,}82 × 6 = 2,533 \newline
    4 курс: 5 дисц. × 93 студ. × 0{,}82 × 6 = 2,287
    & 11,634 \\
    \hline
    Отчёты по практикам & 
    1 курс: 2 × 160 × 0{,}82 = 262 \newline
    2 курс: 1 × 117 × 0{,}82 = 95 \newline
    3 курс: 2 × 103 × 0{,}82 = 168 \newline
    4 курс: 1 × 93 × 0{,}82 = 76 
    & 601 \\
    \hline
    Выпускные квалификационные работы & 
    93 студентов × 0{,}82 
    & 76 \\
    \hline
    \textbf{ИТОГО} & & \textbf{15,394} \\
    \hline
    \end{tabular}
\end{table}
    
\clearpage

\subsubsection{Корректировка оценки с учетом специфики учебного процесса}

Необходимо отметить, что данный анализ основан на ряде допущений:

\begin{itemize}[wide=12.5mm, leftmargin=12.5mm]
    \item Количество гуманитарных дисциплин по курсам и число отчетов на каждой дисциплине может варьироваться в зависимости от требований конкретных преподавателей.
    \item Процент преподавателей, использующих презентации на технических дисциплинах (60\%), является приблизительной оценкой.
    \item Коэффициент активных студентов (0.82) представляет собой усредненное значение и может различаться по курсам и семестрам.
    \item Количество лабораторных работ на дисциплину является средним значением.
\end{itemize}

Для получения более реалистичной оценки необходимо также учесть следующие факторы:

\begin{itemize}[wide=12.5mm, leftmargin=12.5mm]
    \item Не все преподаватели сразу перейдут на использование новой системы.
    \item Часть лабораторных работ может выполняться в виде кода без оформления формального отчета.
    \item Некоторые студенты могут использовать систему не для всех типов работ.
\end{itemize}

С учетом этих факторов, при коэффициенте внедрения системы 0.7 в первый год, можно ожидать накопление около 10,800 документов различных типов. 
При полноценном внедрении системы в образовательный процесс в течение 2-3 лет этот показатель может приблизиться к расчетному значению.

\clearpage
\subsubsection{Качественная оценка потенциальных датасетов}

При внедрении разработанного программно-алгоритмического комплекса возможно формирование следующих типов датасетов:

\begin{enumerate}[wide=12.5mm, leftmargin=12.5mm]
    \item \textbf{Датасеты для анализа научно-технических текстов:}
    \begin{itemize}[wide=12.5mm, leftmargin=12.5mm]
        \item Корпус технической документации на русском языке.
        \item Специализированные терминологические словари по программной инженерии.
        \item Наборы данных для автоматизированной проверки текстов на плагиат.
    \end{itemize}
    \item \textbf{Датасеты для анализа схем и диаграмм:}
    \begin{itemize}[wide=12.5mm, leftmargin=12.5mm]
        \item Коллекции UML-диаграмм различных типов.
        \item Наборы блок-схем алгоритмов.
        \item Схемы баз данных.
    \end{itemize}
    \item \textbf{Датасеты для обучения системам автоматической оценки работ:}
    \begin{itemize}[wide=12.5mm, leftmargin=12.5mm]
        \item Пары <<отчет-оценка>> для предсказания качества работы.
        \item Коллекции типичных ошибок в студенческих работах.
    \end{itemize}
    \item \textbf{Образовательные датасеты:}
    \begin{itemize}[wide=12.5mm, leftmargin=12.5mm]
        \item Структурированные материалы для обучения программированию.
        \item Примеры кода с аннотациями и объяснениями.
    \end{itemize}
\end{enumerate}

Собранные в процессе функционирования системы данные могут быть использованы для:

\begin{itemize}[wide=12.5mm, leftmargin=12.5mm]
\item Развития методов автоматической генерации документации.
\item Исследований в области распознавания диаграмм и схем.
\item Разработки и обучения моделей автоматической проверки кода и технической документации.
\item Создания интеллектуальных систем поддержки образовательного процесса.
\item Анализа паттернов студенческих ошибок и оптимизации учебных программ.
\end{itemize}

Таким образом, внедрение разработанного программно-алгоритмического комплекса не только решит задачи организации совместной работы над проектами, но и создаст ценную инфраструктуру для формирования уникальных датасетов, необходимых для современных исследований в области искусственного интеллекта и обработки данных.

\section{Исследование характеристик разработанного комплекса}

Для оценки производительности разработанной системы было проведено исследование нагрузочных характеристик с использованием инструмента Artillery. 

Технические характеристики машины, на которой производились исследования:

\begin{itemize}[wide=12.5mm, leftmargin=12.5mm]
    \item операционная система: macOS Sequoia 15.4~\cite{sequoia};
    \item оперативная память: 36 Гб;
    \item процессор: Apple M3 Pro~\cite{m3};
    \item количество ядер: 12.
\end{itemize}

В рамках исследования имитировались действия пользователей, включая:
\begin{enumerate}[wide=12.5mm, leftmargin=12.5mm]
    \item регистрацию и вход в систему;
    \item создание файла в проекте;
    \item получение файла по ID;
    \item создание блока контента.
\end{enumerate}
Замеры проводились по метрике среднего времени отклика на каждую из операций. 
Количество одновременных пользователей варьировалось от 10 до 200. 
Были выбраны следующие значения нагрузки: 10, 15, 20, 30, 40, 50, 70, 90, 100, 140, 170 и 200 пользователей.

На рисунке~\ref{img:mean_time} представлена более подробная диаграмма компонентов.

\includeimage
	{mean_time}
	{f}
	{H}
	{1\textwidth}
	{Зависимость среднего времени ответа от числа пользователей}

График показывает, что среднее время отклика системы увеличивается с ростом числа пользователей, однако этот рост происходит нелинейно и приближен к логарифмическому типу зависимости. 
При 20-кратном увеличении нагрузки (с 10 до 200 пользователей) среднее время отклика увеличилось только на 4.2 мс (с 18.3 до 22.5 мс), что составляет примерно 23\%.
Это свидетельствует о достаточной устойчивости и масштабируемости архитектуры при увеличении нагрузки.

\section*{Вывод}

Система обладает значительным потенциалом для формирования структурированных датасетов — при внедрении в образовательный процесс кафедры может быть накоплено около 15 000 документов в год, что представляет ценность для исследований в области искусственного интеллекта. 
Исследование нагрузочных характеристик продемонстрировало масштабируемость системы: при 20-кратном увеличении нагрузки время отклика выросло лишь на 23\%, что свидетельствует об оптимальности выбранных архитектурных решений. 
Логарифмический характер зависимости времени отклика от числа пользователей гарантирует стабильную работу даже при значительном росте пользовательской базы.