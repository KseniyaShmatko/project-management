\chapter*{ПРИЛОЖЕНИЕ А}
	\addcontentsline{toc}{chapter}{ПРИЛОЖЕНИЕ А}
	\renewcommand{\thelstlisting}{А.\arabic{lstlisting}}
	\setcounter{lstlisting}{0}
	В листингах~\ref{lst:Project_m.kt}--\ref{lst:TaskColumn.kt} представлена реализация моделей в серверной части приложения. 
	\includelisting
		{Project_m.kt}
		{Модель проекта}
	\includelisting
		{ProjectFile.kt}
		{Модель связи проекта и файлов}
	\clearpage
	\includelisting
		{File.kt}
		{Модель файла}
	\clearpage
	\includelisting
		{User.kt}
		{Модель пользователя}
	\clearpage
	\includelisting
		{ProjectUser.kt}
		{Модель связи проекта и пользователя}
	\includelisting
		{ContentBlock.kt}
		{Модель блока данных}
	\clearpage
	\includelisting
		{SuperObject.kt}
		{Модель суперобъекта}
	\includelisting
		{Style.kt}
		{Модель стилей}
	\includelisting
		{StylesMap.kt}
		{Модель связи стилей и блоков}	
	\clearpage
	\includelisting
		{SchemeNode.kt}
		{Модель блока схемы}
	\includelisting
		{SchemeEdge.kt}
		{Модель связей блоков схемы}	
	\includelisting
		{CalendarEvent.kt}
		{Модель события календаря}
	\clearpage
	\includelisting
		{TaskItem.kt}
		{Модель задачи}
	\includelisting
		{TaskColumn.kt}
		{Модель колонки задач}	
\chapter*{ПРИЛОЖЕНИЕ Б}
	\addcontentsline{toc}{chapter}{ПРИЛОЖЕНИЕ Б}
	\renewcommand{\thelstlisting}{Б.\arabic{lstlisting}}
	\setcounter{lstlisting}{0}
	В листингах~\ref{lst:EditorJsWrapper.tsx}--\ref{lst:EditorJsWrapper5.tsx} представлена реализация редактора в клиентской части приложения. 
	\includelisting
		{EditorJsWrapper.tsx}
		{Редактор заметки часть 1}
	\clearpage
	\includelisting
		{EditorJsWrapper2.tsx}
		{Редактор заметки часть 2}
	\clearpage
	\includelisting
		{EditorJsWrapper3.tsx}
		{Редактор заметки часть 3}
	\clearpage
	\includelisting
		{EditorJsWrapper4.tsx}
		{Редактор заметки часть 4}
	\clearpage
	\includelisting
		{EditorJsWrapper5.tsx}
		{Редактор заметки часть 5}
\chapter*{ПРИЛОЖЕНИЕ В\\}
	\addcontentsline{toc}{chapter}{ПРИЛОЖЕНИЕ В}
	\renewcommand{\thefigure}{В.\arabic{figure}}
	\setcounter{figure}{0}
	На рисунках~\ref{img:login}--\ref{img:roles} представлен интерфейс приложения.
	\includeimage
		{login}
		{f}
		{H}
		{1\textwidth}
		{Экран входа в систему}
	\includeimage
		{auth}
		{f}
		{H}
		{1\textwidth}
		{Экран регистрации в системе}
	\includeimage
		{project}
		{f}
		{H}
		{1\textwidth}
		{Экрана выбора проекта и просмотра файлов}
	\includeimage
		{note}
		{f}
		{H}
		{1\textwidth}
		{Экрана редактирования заметки}
	\includeimage
		{roles}
		{f}
		{H}
		{1\textwidth}
		{Экрана управления ролями в проекте}
\chapter*{ПРИЛОЖЕНИЕ Г}
	\addcontentsline{toc}{chapter}{ПРИЛОЖЕНИЕ Г}
	\renewcommand{\thelstlisting}{В.\arabic{lstlisting}}
	\setcounter{lstlisting}{0}
	Презентация к выпускной квалификационной работе состоит из 17 слайдов.