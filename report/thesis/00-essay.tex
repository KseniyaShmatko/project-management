\begin{essay}{}
    В данной работе рассматривается проектирование и разработка программно-алгоритмического комплекса с многоцелевой и масштабируемой архитектурой, предназначенного для организации совместной работы пользователей и управления проектами.

    Проведен анализ предметной области, включающий обзор и сравнение современных реляционных и нереляционных систем хранения данных. 
    Рассмотрены характеристики распределенных систем хранения и методы организации сетевого многопользовательского взаимодействия.
    
    Разработаны основные положения и структура предлагаемого программно-алгоритмического комплекса.
    Описана его модульная архитектура, ключевые компоненты и принципы их взаимодействия. 
    Определены основные структуры данных, используемые для хранения пользовательской информации, проектных данных и создаваемого контента.
    
    Обоснован выбор средств программной реализации. 
    Выполнено тестирование разработанного комплекса для проверки корректности его работы. 
    Описаны основные сценарии взаимодействия пользователя с программным обеспечением.
    
    Проведено исследование потенциальной применимости разработанного комплекса в различных сценариях управления проектами и командной работы. 
    Исследованы ключевые характеристики системы.
    
    Ключевые слова: совместная работа, управление проектами, программный комплекс, масштабируемая архитектура, распределенное хранение данных, MongoDB, PostgreSQL, WebSockets, многопользовательское взаимодействие, гибридная база данных.
\end{essay}