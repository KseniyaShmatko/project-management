\chapter*{ЗАКЛЮЧЕНИЕ}
\addcontentsline{toc}{chapter}{ЗАКЛЮЧЕНИЕ}

В ходе выполнения выпускной квалификационной работы был разработан программно-алгоритмический комплекс с многоцелевой и масштабируемой архитектурой, предназначенный для совместной работы и управления проектами. 
Разработка велась с учетом потребности в универсальных платформах, способных обеспечить комплексную поддержку командной деятельности в условиях современных форматов работы. 
Особое внимание было уделено возможности хранения и управления неоднородными данными, генерируемыми пользователями, что открывает перспективы для их дальнейшего использования в исследовательских целях.

В соответствии с поставленной целью и задачами работы было выполнено следующее:
\begin{enumerate}[wide=12.5mm, leftmargin=12.5mm]
    \item Были рассмотрены современные системы хранения данных, включая их классификацию и особенности работы с различными типами данных, проанализированы методы сетевого многопользовательского взаимодействия, сформулированы и формализованы исходные задачи и требования к разрабатываемому комплексу.
    \item Предложена многоуровневая архитектура серверного приложения (контроллеры, сервисы, слой доступа к данным) и структура клиентского SPA-приложения, описаны основные компоненты системы, их взаимодействие, проработаны ключевые структуры данных с использованием гибридной модели хранения: PostgreSQL для реляционных метаданных и MongoDB для хранения контента.
    \item Выбран технологический стек, включающий Kotlin и Spring Boot для серверной части, TypeScript и React для клиентской, разработаны основные функциональные модули системы, проведено функциональное тестирование реализованных модулей и представлены примеры модульных тестов для серверной части, описано взаимодействие пользователя с программным обеспечением.
    \item Было проведено исследование в ходе которого было определено, что реализованный функционал соответствует основным поставленным задачам и обеспечивает базовые возможности для совместной работы, заложенная архитектура и модель данных создают основу для дальнейшего расширения и добавления новых типов сервисов, система имеет значительный потенциал для формирования структурированных датасетов.
\end{enumerate}

К достоинствам разработанного программно-алгоритмического комплекса можно отнести:
\begin{itemize}[wide=12.5mm, leftmargin=12.5mm]
    \item Гибридность модели хранения и организацию моделей, позволяющих управлять разнообразными типами контента и расширять систему новыми сервисами.
    \item Модульность и расширяемость на стороне сервера и на стороне клиента.
    \item Формирование наборов данных, которые могут быть использованы для дальнейших исследований.
\end{itemize}

К недостаткам и областям для дальнейшего развития относятся:
\begin{itemize}[wide=12.5mm, leftmargin=12.5mm]
    \item Ограниченный набор реализованных сервисов.
    \item Отсутствие полнофункционального механизма совместного редактирования в реальном времени для всех пользователей.
    \item Необходимость развертывания и интеграции комплекса на серверной инфраструктуре университета.
\end{itemize}

Поставленная цель работы --- разработка программно-алгоритмического комплекса с многоцелевой и масштабируемой архитектурой для совместной работы и управления проектами --- была достигнута. 
Разработанное программное обеспечение решает основные поставленные задачи, закладывает фундамент для дальнейшего развития и может служить основой для создания полнофункциональной платформы для командной работы, а также для формирования исследовательских наборов данных.
