\begin{definitions}
	\definition{Системы хранения данных}{cовокупность языковых и программных средств, предназначенных для создания, ведения и совместного использования БД многими пользователями.}
	\definition{Распределенная система хранения данных}{система хранения, в которой данные физически расположены на нескольких узлах (серверах), связанных сетью, но представляются пользователю или приложению как единое целое.} 
	\definition{Масштабируемость}{способность системы, сети или процесса увеличивать свою производительность и пропускную способность пропорционально увеличению нагрузки (количества пользователей, объема данных, транзакций) путем добавления ресурсов (аппаратных или программных).}
	\definition{PostgreSQL}{объектно-реляционная система управления базами данных, соответствующая стандарту SQL, расширяемая и поддерживающая сложные запросы и транзакции (ACID).}
	\definition{MongoDB}{документоориентированная NoSQL система управления базами данных, использующая для хранения данных JSON-подобные документы (BSON).}
	\definition{WebSocket}{сетевой протокол, обеспечивающий постоянное полнодуплексное (двустороннее) соединение между клиентом и сервером поверх одного TCP-соединения, предназначенный для интерактивного обмена данными в реальном времени.}
	\definition{Mind map (интеллект-карта, диаграмма связей)}{метод структуризации и визуализации концепций с использованием графической записи в виде диаграммы.}
\end{definitions}