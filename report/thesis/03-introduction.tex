\chapter*{ВВЕДЕНИЕ}
\addcontentsline{toc}{chapter}{ВВЕДЕНИЕ}

В условиях динамично развивающейся цифровой экономики и роста популярности удаленных и гибридных форматов работы, организация совместной деятельности и управления проектами становится критически важной для успеха команд и организаций. 
Существующие инструменты часто либо специализированы на узком круге задач, либо представляют собой разрозненные сервисы, что усложняет интеграцию рабочих процессов и управление информацией. 
Возникает потребность в универсальных платформах, обеспечивающих комплексную поддержку командной работы.
Более того, данные, генерируемые пользователями в рамках работы с подобным комплексом (текстовые записки, созданные схемы, файлы проектов, история коммуникаций в чатах и др.), могут послужить основой для формирования уникальных наборов данных (датасетов). 
Эти датасеты в перспективе могут быть использованы для проведения исследований и выполнения выпускных квалификационных работ студентами последующих курсов.

Целью данной работы является разработка программно-алгоритмического комплекса с многоцелевой и масштабируемой архитектурой для совместной работы и управления проектами.

Для достижения поставленной цели необходимо решить следующие задачи:

\begin{enumerate}[wide=12.5mm, leftmargin=12.5mm]
    \item Провести анализ предметной области, обзор и сравнение существующих систем хранения данных и методов сетевого многопользовательского взаимодействия, а также формализовать требования к разрабатываемому комплексу.
    \item Разработать архитектуру программно-алгоритмического комплекса, структуру приложения, описание его компонентов, их взаимодействия и ключевых структур данных.
    \item Обосновать выбор средств программной реализации, разработать программное обеспечение, реализующее основные функции комплекса, и выполнить его тестирование для проверки корректности работы.
    \item Провести исследование применимости и оценить ключевые характеристики разработанного программно-алгоритмического комплекса.
\end{enumerate}