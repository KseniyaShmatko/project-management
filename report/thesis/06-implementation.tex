\chapter{Технологический раздел}

\section{Выбор средств программной реализации}

При выборе технологий для разработки программно-алгоритмического комплекса основной упор делался на их способность обеспечить реализацию поставленных задач, а также на распространенность инструментов и наличие документации.

Язык Kotlin был выбран в качестве основного для серверной разработки, так как он предоставляет все необходимые средства для решения поставленных задач. 
Ключевые аспекты выбора:
\begin{itemize}[wide=12.5mm, leftmargin=12.5mm]
    \item Стандартная библиотека и возможности языка (например, корутины для асинхронных операций, data-классы) покрывают широкий спектр задач, возникающих при разработке.
    \item Возможность использования существующих Java-библиотек и фреймворков, включая Spring.
\end{itemize}

Фреймворк Spring Boot используется для построения REST API, интеграции с базами данных PostgreSQL и MongoDB, а также для реализации механизмов безопасности через Spring Security с JWT. 
Данный фреймворк предоставляет необходимую инфраструктуру для обработки HTTP-запросов, управления транзакциями и конфигурацией приложения.

TypeScript выбран для разработки клиентской части, так как статическая типизация способствует более предсказуемой разработке интерфейсов со сложной логикой. 
Библиотека React позволяет:
\begin{itemize}[wide=12.5mm, leftmargin=12.5mm]
    \item Структурировать интерфейс в виде переиспользуемых компонентов.
    \item Оптимизированно обновлять DOM за счет использования концепции виртуального DOM.
\end{itemize}

Для управления общим состоянием, таким как информация об аутентифицированном пользователе, применяется React Context API. 
Маршрутизация в одностраничном приложении реализована с помощью библиотеки React Router.
Для выполнения HTTP-запросов к серверному REST API применяется библиотека axios.

\section{Тестирование программно-алгоритмического комплекса}

Для проверки корректности работы реализованных модулей проводилось сквозное тестирование, а также были разработаны модульные тесты для некоторых компонентов серверной части.

В ходе разработки выполнялось тестирование ключевых пользовательских сценариев, таких как:
\begin{itemize}[wide=12.5mm, leftmargin=12.5mm]
    \item Регистрация и аутентификация пользователя.
    \item Создание и выбор проекта.
    \item Создание заметки.
    \item Добавление и базовое редактирование контента в заметке.
    \item Загрузка изображения в заметку.
    \item Добавление и удаление участников проекта, изменение их роли.
\end{itemize}
В листингах~\ref{lst:auth.kt} и~\ref{lst:project.kt} представлен E2E тест для сценария создания заметки, включая:  
\begin{itemize}[wide=12.5mm, leftmargin=12.5mm]
    \item регистрацию пользователя;
    \item авторизацию и получение JWT;
    \item создание проекта;
    \item создание типа файла;
    \item создание файла с привязкой к проекту.
\end{itemize}
\clearpage
\includelisting
	{e2e.kt}
	{E2E тест для сценария создания заметки (часть 1)}
\clearpage
\includelisting
    {e2e2.kt}
    {E2E тест для сценария создания заметки (часть 2)}
\clearpage
\includelisting
    {e2e3.kt}
    {E2E тест для сценария создания заметки (часть 3)}

Было проведено модульное тестированию серверной части. 
В листингах~\ref{lst:auth.kt} и~\ref{lst:project.kt} представлены тесты регистрации пользователя и создания проекта. 

\clearpage
\includelisting
	{auth.kt}
	{Тест для AuthController (регистрация)}

\clearpage
\includelisting
	{project.kt}
	{Тест для ProjectService (создание проекта)}

\clearpage
Реализация серверной и клиентской части приложения представлены в Приложении Б и В соответственно.

\section{Описание взаимодействия пользователя с программным обеспечением}

Основные сценарии работы с реализованным функционалом (проекты и заметки).

\begin{enumerate}[wide=12.5mm, leftmargin=12.5mm]
    \item \textbf{Вход и регистрация.}
    Пользователь начинает работу со страницы входа. Если у пользователя нет учетной записи, он может перейти на страницу регистрации. Интерфейсы этих страниц стандартны и включают поля для ввода логина/e-mail, пароля, имени и фамилии (при регистрации). После успешной аутентификации пользователь перенаправляется на главную страницу приложения — панель управления проектами.
    \item \textbf{Работа с проектами.}
    На панели управления проектами интерфейс разделен на боковую панель для навигации по проектам и основную рабочую область для отображения содержимого выбранного проекта.
    Пользователь может:
    \begin{itemize}[wide=12.5mm, leftmargin=12.5mm]
        \item Создать новый проект, указав его название в модальном окне.
        \item Выбрать существующий проект из списка на боковой панели.
        \item В основной области просматривать файлы выбранного проекта.
        \item Создать новый файл (заметку) в текущем проекте.
        \item Редактировать названия проектов и файлов.
        \item Управлять доступом к проекту (для владельца): добавлять других пользователей по логину, назначать им роли (<<Редактор>>, <<Читатель>>) и удалять их из проекта через модальное окно.
    \end{itemize}
    \item \textbf{Редактирование заметки.} Пользователь может:
    \begin{itemize}[wide=12.5mm, leftmargin=12.5mm]
        \item Вводить и форматировать текст.
        \item Вставлять, удалять, перемещать и изменять различные типы блоков: заголовки, списки, цитаты, изображения.
    \end{itemize}
\end{enumerate}

Изменения в документе сохраняются автоматически с определенным интервалом, отправляя данные на сервер для синхронизации. Пользователи, не имеющие прав на редактирование, видят документ в режиме «только чтение».
Скриншоты основных экранов интерфейса представлены в Приложении Г.

\section{Описание API программно-алгоритмического комплекса}

Разработанный программно-алгоритмический комплекс предоставляет REST API для взаимодействия с клиентскими приложениями. 
API построено в соответствии с принципами REST и организовано по функциональным группам для обеспечения логического разделения интерфейсов доступа к различным сущностям системы.

\subsubsection{Основные группы API}

\begin{enumerate}[wide=12.5mm, leftmargin=12.5mm]
    \item \textbf{Аутентификация} — интерфейсы для регистрации и аутентификации пользователей:
    \begin{itemize}[wide=12.5mm, leftmargin=12.5mm]
        \item \texttt{POST /users/login} — аутентификация пользователя и получение JWT-токена;
        \item \texttt{POST /users/register} — регистрация нового пользователя.
    \end{itemize}

    \item \textbf{Управление пользователями} — операции с данными пользователей:
    \begin{itemize}[wide=12.5mm, leftmargin=12.5mm]
        \item \texttt{GET /users/me} — получение информации о текущем аутентифицированном пользователе;
        \item \texttt{GET /users/search} — поиск пользователей по логину (для добавления в проекты);
        \item \texttt{GET|PUT|DELETE /users/\{user\_id\}} — получение, обновление и удаление пользователя.
    \end{itemize}

    \item \textbf{Управление проектами} — создание и управление проектами:
    \begin{itemize}[wide=12.5mm, leftmargin=12.5mm]
        \item \texttt{POST /projects} — создание нового проекта;
        \item \texttt{GET /projects} — получение списка проектов текущего пользователя;
        \item \texttt{GET|PUT|DELETE /projects/\{project\_id\}} — получение, обновление и удаление проекта.
    \end{itemize}

    \item \textbf{Управление доступом к проектам} — настройка прав доступа:
    \begin{itemize}[wide=12.5mm, leftmargin=12.5mm]
        \item \texttt{POST /projects-users} — добавление пользователя в проект с указанием роли;
        \item \texttt{GET /projects-users/project/\{project\_id\}/users} — получение списка пользователей проекта;
        \item \texttt{PUT /projects-users/project/\{project\_id\} /user/\{user\_id\}} — изменение роли пользователя;
        \item \texttt{DELETE /projects-users/project/\{project\_id\} /user/\{user\_id\}} — удаление пользователя из проекта.
    \end{itemize}

    \item \textbf{Управление файлами в проектах} — операции с файлами в контексте проектов:
    \begin{itemize}[wide=12.5mm, leftmargin=12.5mm]
        \item \texttt{GET /projects/\{project\_id\}/files} — получение всех файлов проекта;
        \item \texttt{POST /projects/\{project\_id\}/files} — загрузка файла в проект;
        \item \texttt{GET|DELETE /projects/\{project\_id\}/files/\{file\_id\}} — получение и удаление файла из проекта;
        \item \texttt{PATCH /projects/\{project\_id\}/files/\{file\_id\}/name} — обновление имени файла;
        \item \texttt{PATCH /projects/\{project\_id\}/files/ \{file\_id\}/super-object} — связывание файла с документом;
        \item \texttt{GET /projects/\{project\_id\}/files/\{file\_id\}/download} — скачивание файла.
    \end{itemize}
    \pagebreak
    \item \textbf{Хранение документов (суперобъект)} — работа со структурой документов:
    \begin{itemize}[wide=12.5mm, leftmargin=12.5mm]
        \item \texttt{POST /super-objects} — создание нового документа;
        \item \texttt{GET /super-objects/by-file/\{fileId\}} — получение документа по идентификатору файла;
        \item \texttt{GET|PUT|DELETE /super-objects/\{id\}} — получение, обновление и удаление документа;
        \item \texttt{PUT /super-objects/\{superObjectId\}/sync-blocks} — синхронизация блоков документа.
    \end{itemize}

    \item \textbf{Блоки контента} — управление блоками содержимого документов:
    \begin{itemize}[wide=12.5mm, leftmargin=12.5mm]
        \item \texttt{POST /content-blocks} — создание нового блока;
        \item \texttt{GET|PUT|DELETE /content-blocks/\{id\}} — получение, обновление и удаление блока.
    \end{itemize}

    \item \textbf{Стили} и \textbf{карты стилей} — управление форматированием документов:
    \begin{itemize}[wide=12.5mm, leftmargin=12.5mm]
        \item \texttt{POST /styles}, \texttt{POST /styles-maps} — создание стиля и карты стилей;
        \item \texttt{GET|PUT|DELETE /styles/\{id\}} — операции со стилями;
        \item \texttt{GET|PUT|DELETE /styles-maps/\{id\}} — операции с картой стилей.
    \end{itemize}
\end{enumerate}

\subsubsection{Особенности реализации API}

API реализовано с использованием стандартных HTTP-методов:
\begin{itemize}[wide=12.5mm, leftmargin=12.5mm]
    \item \texttt{GET} — для получения данных;
    \item \texttt{POST} — для создания новых ресурсов;
    \item \texttt{PUT} — для полного обновления ресурсов;
    \item \texttt{PATCH} — для частичного обновления ресурсов;
    \item \texttt{DELETE} — для удаления ресурсов.
\end{itemize}

Ответы сервера включают соответствующие HTTP-коды состояния:
\begin{itemize}[wide=12.5mm, leftmargin=12.5mm]
    \item 200 (OK) — успешное выполнение операции;
    \item 201 (Created) — успешное создание ресурса;
    \item 204 (No Content) — успешное выполнение операции без возвращаемых данных;
    \item 400 (Bad Request) — ошибка в запросе;
    \item 401 (Unauthorized) — отсутствие аутентификации;
    \item 404 (Not Found) — запрашиваемый ресурс не найден.
\end{itemize}

Безопасность API обеспечивается через JWT-токены, передаваемые в заголовке \texttt{Authorization} каждого запроса, требующего аутентификации.
Авторизация для доступа к ресурсам проверяется на серверной стороне на основе ролей пользователя в проекте (владелец, редактор, наблюдатель).

Данный набор API охватывает все необходимые операции для работы с программно-алгоритмическим комплексом и обеспечивает полноценное взаимодействие между клиентской и серверной частями приложения.
Также для облегчения взаимодействия с API и его документирования был интегрирован Swagger UI. 
Спецификация OpenAPI содержит подробное описание всех эндпоинтов, включая требуемые параметры, форматы запросов и ответов, возможные коды ошибок и примеры использования. 

\section*{Вывод}

В данном разделе был обоснован выбор основных средств разработки: Kotlin и Spring Boot для серверной части, TypeScript и React для клиентской. 
Представлены подходы к тестированию, включая примеры модульных тестов для серверной части. 
Описаны основные сценарии взаимодействия пользователя с реализованным функционалом и предоставлено описание API.
Разработанный программный комплекс является основой для дальнейшего расширения и добавления новых сервисов.
