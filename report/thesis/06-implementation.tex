\chapter{Технологический раздел}

\section{Выбор средств программной реализации}

При выборе технологий для разработки программно-алгоритмического комплекса основной упор делался на их способность обеспечить реализацию поставленных задач, а также на распространенность инструментов и наличие документации.

Язык Kotlin был выбран в качестве основного для серверной разработки, так как он предоставляет все необходимые средства для решения поставленных задач. 
Ключевые аспекты выбора:
\begin{itemize}
    \item Стандартная библиотека и возможности языка (например, корутины для асинхронных операций, data-классы) покрывают широкий спектр задач, возникающих при разработке.
    \item Возможность использования существующих Java-библиотек и фреймворков, включая Spring.
\end{itemize}

Фреймворк Spring Boot используется для построения REST API, интеграции с базами данных PostgreSQL и MongoDB, а также для реализации механизмов безопасности через Spring Security с JWT. 
Данный фреймворк предоставляет необходимую инфраструктуру для обработки HTTP-запросов, управления транзакциями и конфигурацией приложения.

TypeScript выбран для разработки клиентской части, так как статическая типизация способствует более предсказуемой разработке интерфейсов со сложной логикой. 
Библиотека React позволяет:
\begin{itemize}
    \item Структурировать интерфейс в виде переиспользуемых компонентов.
    \item Оптимизированно обновлять DOM за счет использования концепции виртуального DOM.
\end{itemize}

Для управления общим состоянием, таким как информация об аутентифицированном пользователе, применяется React Context API. 
Маршрутизация в одностраничном приложении реализована с помощью библиотеки React Router.
Для выполнения HTTP-запросов к серверному REST API применяется библиотека axios.

\section{Тестирование программно-алгоритмического комплекса}

Для проверки корректности работы реализованных модулей проводилось функциональное тестирование, а также были разработаны модульные тесты для некоторых компонентов серверной части.

В ходе разработки выполнялось ручное тестирование ключевых пользовательских сценариев, таких как:
\begin{itemize}
    \item Регистрация и аутентификация пользователя.
    \item Создание и выбор проекта.
    \item Создание, открытие и переименование заметки.
    \item Добавление и базовое редактирование контента в заметке.
    \item Загрузка изображения в заметку.
    \item Добавление и удаление участников проекта, изменение их роли.
\end{itemize}
Тестирование API эндпоинтов проводилось с использованием инструмента Postman.

Было проведено модульное тестированию серверной части. 
В листингах~\ref{lst:auth.kt} и~\ref{lst:project.kt} представлены тесты регистрации пользователя и создания проекта. 

\clearpage
\includelisting
	{auth.kt}
	{Тест для AuthController (регистрация)}

\clearpage
\includelisting
	{project.kt}
	{Тест для ProjectService (создание проекта)}

\clearpage
Реализация серверной и клиентской части приложения представлены в Приложении А и Б соответственно.

\section{Описание взаимодействия пользователя с программным обеспечением}

Взаимодействие пользователя с комплексом осуществляется через веб-интерфейс, реализованный как одностраничное приложение (SPA). 
Далее описаны основные сценарии работы с реализованным функционалом (проекты и заметки).

\begin{enumerate}
    \item \textbf{Вход и регистрация:}
    Пользователь начинает работу со страницы входа. Если у пользователя нет учетной записи, он может перейти на страницу регистрации. Интерфейсы этих страниц стандартны и включают поля для ввода логина/e-mail, пароля, имени и фамилии (при регистрации). После успешной аутентификации пользователь перенаправляется на главную страницу приложения — панель управления проектами.
    \item \textbf{Работа с проектами:}
    На панели управления проектами интерфейс разделен на боковую панель для навигации по проектам и основную рабочую область для отображения содержимого выбранного проекта.
    Пользователь может:
    \begin{itemize}
        \item Создать новый проект, указав его название в модальном окне.
        \item Выбрать существующий проект из списка на боковой панели.
        \item В основной области просматривать файлы выбранного проекта.
        \item Создать новый файл (заметку) в текущем проекте.
        \item Редактировать названия проектов и файлов.
        \item Управлять доступом к проекту (для владельца): добавлять других пользователей по логину, назначать им роли (<<Редактор>>, <<Читатель>>) и удалять их из проекта через модальное окно.
    \end{itemize}
    \item \textbf{Редактирование заметки:}
    При открытии файла-заметки пользователь переходит на страницу редактора. Пользователь может:
    \begin{itemize}
        \item Вводить и форматировать текст.
        \item Вставлять различные типы блоков: заголовки, списки, цитаты, изображения.
        \item Удалять, перемещать и изменять блоки данных.
    \end{itemize}
    Изменения в документе сохраняются автоматически с определенным интервалом, отправляя данные на сервер для синхронизации. Пользователи, не имеющие прав на редактирование, видят документ в режиме «только чтение».
\end{enumerate}
Скриншоты основных экранов интерфейса представлены в Приложении В.

\section*{Вывод}

В данном разделе был обоснован выбор основных средств разработки: Kotlin и Spring Boot для серверной части, TypeScript и React для клиентской. 
Представлены подходы к тестированию, включая примеры модульных тестов для серверной части. 
Описаны основные сценарии взаимодействия пользователя с реализованным функционалом.
Разработанный программный комплекс является основой для дальнейшего расширения и добавления новых сервисов.
