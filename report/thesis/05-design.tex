\chapter{Конструкторский раздел}

\section{Основные положения предлагаемого программно-алгоритмического комплекса}

При проектировании программно-алгоритмического комплекса были заложены следующие основные положения, направленные на обеспечение его функциональности, масштабируемости и многоцелевого использования:

\begin{enumerate}[wide=12.5mm, leftmargin=12.5mm]
    \item \textbf{Клиент-серверная архитектура:} Комплекс реализуется в рамках клиент-серверной архитектуры, где пользователь взаимодействует с системой через клиентское приложение (веб-браузер), а основная бизнес-логика, управление данными и синхронизация выполняются на серверной стороне.
    \item \textbf{Гибридная модель хранения данных:} Учитывая необходимость работы с неоднородными данными, применяется гибридный подход к хранению:
        \begin{itemize}
            \item \textbf{PostgreSQL} используется для хранения структурированных, реляционных данных, требующих строгой схемы, целостности и поддержки транзакций. Сюда относятся данные о пользователях, проектах, файлах (метаданные), типах файлов, а также связи между ними (участие пользователей в проектах, принадлежность файлов проектам, роли и права доступа).
            \item \textbf{MongoDB} используется для хранения полуструктурированных и неструктурированных данных, требующих гибкости схемы и горизонтальной масштабируемости. В MongoDB хранится фактическое содержимое файлов, состоящее из различных блоков данных, а также информация о стилях.
        \end{itemize}
    \item \textbf{Комбинированное использование REST API и WebSockets:} Для взаимодействия между клиентом и сервером применяются два основных механизма:
        \begin{itemize}
            \item \textbf{REST API} используется для выполнения стандартных CRUD-операций (создание, чтение, обновление, удаление) над основными сущностями системы (пользователи, проекты, файлы и т.д.), а также для запроса или модификации данных, не требующих немедленной синхронизации у других пользователей.
            \item \textbf{WebSocket} используется для обеспечения взаимодействия в реальном времени важного для функций совместной работы: одновременное редактирование документов, обмен сообщениями в чате, доставка мгновенных уведомлений, отображение статуса присутствия пользователей.
        \end{itemize}
    \item \textbf{Управление неоднородными данными:} Для управления разнообразным контентом (текстовые заметки, схемы, презентации, задачи и т.д.) каждый документ будет представлен как совокупность таких блоков различных типов (например, параграф, заголовок, изображение, узел схемы, задача). Центральным элементом этой модели является "суперобъект" (`SuperObject`), хранящийся в MongoDB. Он выступает в роли контейнера метаинформации о файле и обеспечивает связь с набором этих блоков или со специфической структурой данных, соответствующей определенному типу сервиса (например, коллекция узлов и ребер для схемы, события для календаря). Такой подход позволяет добавлять новые типы контента и управлять документами с различной внутренней структурой в рамках единого механизма.
    \item \textbf{Модульность серверного приложения:} Серверное приложение проектируется с учетом модульности, где каждая основная функциональная область (управление пользователями, проектами, контентом и т.д.) выделяется в логический компонент, что упрощает разработку, тестирование и дальнейшее развитие системы.
    \item \textbf{Масштабируемость:} Архитектура и выбор технологий (в частности, использование MongoDB и возможность репликации/масштабирования PostgreSQL) закладывают основу для потенциальной горизонтальной и вертикальной масштабируемости системы при росте нагрузки.
\end{enumerate}

\section{Структура и компоненты программного приложения}

Программно-алгоритмический комплекс строится на основе многоуровневой архитектуры, близкой к принципам чистой или луковой архитектуры, с четким разделением ответственности между слоями.
Общая структура представлена на рисунке~\ref{img:structure-common}.

\includeimage
	{structure-common}
	{f}
	{H}
	{1\textwidth}
	{Общая архитектура программно-алгоритмического комплекса}


Основные компоненты и слои архитектуры:

\begin{enumerate}[wide=12.5mm, leftmargin=12.5mm]
    \item \textbf{Клиентское приложение} будет реализовано как одностраничное веб-приложение (SPA), отвечающее за пользовательский интерфейс, визуализацию данных и взаимодействие с пользователем. Обмен данными с серверной частью будет осуществляется через REST API и WebSocket.
    \item \textbf{Серверное приложение} реализует бизнес-логику и будет состоять из следующих слоев:
        \begin{itemize}
            \item \textbf{Слой Контроллеров}, который отвечает за обработку входящих HTTP-запросов и вызов методов соответствующего сервиса. В этот же слой логически входит обработка WebSocket-соединений для real-time взаимодействия.
            \item \textbf{Слой Сервисов}, который инкапсулирует основную бизнес-логику приложения. Сервисы координируют работу репозиториев, выполняют преобразование данных, реализуют правила предметной области и управляют транзакциями.
            \item \textbf{Слой Доступа к Данным}, включающий интерфейсы репозиториев и абстрагирующий детали взаимодействия с базами данных, предоставляя методы для CRUD-операций.
        \end{itemize}
    \item \textbf{Система хранения реляционных данных} хранит метаданные и связи между основными сущностями системы.
    \item \textbf{Система хранения документных данных:} хранит гибкое содержимое файлов и связанные с ним стили.
\end{enumerate}

На рисунке~\ref{img:components} представлена более подробная диаграмма компонентов серверного приложения.

\includeimage
	{components}
	{f}
	{H}
	{1\textwidth}
	{Диаграмма компонентов серверного приложения}

\clearpage

На рисунке~\ref{img:class} приведена диаграмма классов основных бизнес-сущностей системы и их взаимодействия.
Она фокусируется на ключевых сущностях User (Пользователь) и Project (Проект), а также на компонентах, отвечающих за их обработку (контроллеры, сервисы, репозитории).

\includeimage
	{class}
	{f}
	{H}
	{1\textwidth}
	{Диаграмма классов для сущностей User, Project и связанных компонентов}

\clearpage
Выбор для демонстрации именно этих классов и связанных с ними компонентов обусловлен следующими причинами:
\begin{itemize}
    \item Сущности User и Project являются основополагающими в разрабатываемой системе управления проектами и совместной работы. Практически вся функциональность так или иначе связана с пользователями и проектами, к которым они имеют доступ.
    \item Структура этих классов и паттерны их обработки являются типичными и применяются для большинства других сущностей системы (например, для файлов, типов файлов, блоков контента). Таким образом, данная диаграмма иллюстрирует общий подход к проектированию и организации кода на серверной стороне.
\end{itemize}

\section{Сценарий совместного редактирования}

Для иллюстрации взаимодействия компонентов системы при выполнении операций в реальном времени рассмотрим сценарий совместного редактирования документа. 
\clearpage
Процесс представлен диаграммой последовательности на рисунке~\ref{img:sequence}.

\includeimage
	{sequence}
	{f}
	{H}
	{1.3\textwidth, angle=90}
	{Диаграмма последовательности для совместного редактирования документа}

Основные шаги взаимодействия при внесении изменения пользователем~\textit{1} в документ, который также просматривает пользователь~\textit{2}:

\begin{enumerate}
    \item Пользователь~\textit{1} выполняет действие редактирования (например, вводит символ) в клиентском приложении~\textit{1}.
    \item Клиентское приложение~\textit{1} формирует сообщение об операции редактирования, указывая тип операции, позицию, измененные данные и идентификатор документа/файла, и отправляет его на сервер через установленное WebSocket-соединение.
    \item Серверное приложение, в частности его компонент, отвечающий за обработку WebSocket-сообщений, принимает данное сообщение.
    \item Обработчик WebSocket вызывает соответствующий метод сервисного слоя, ответственного за управление контентом, передавая ему детали операции редактирования.
    \item Сервис контента проверяет полученную операцию и взаимодействует со слоем доступа к данным для сохранения изменений в соответствующей системе хранения.
    \item После подтверждения успешного сохранения данных от слоя доступа к данным, компонент обработки WebSocket-сообщений получает уведомление об успешном применении операции.
    \item Компонент обработки WebSocket-сообщений рассылает информацию о выполненной операции всем остальным пользователям, включая пользователя~\textit{2}, которые в данный момент работают с этим же документом и подключены к соответствующей WebSocket-сессии.
    \item Клиентское приложение~\textit{2} получает сообщение об операции через WebSocket и применяет соответствующие изменения к своему локальному представлению документа, отображая актуальное состояние пользователю~\textit{2}.
\end{enumerate}

Данный механизм обеспечивает синхронизацию состояния документа у всех активных пользователей в режиме реального времени.

\section{Ключевые структуры данных}

Для хранения информации выбраны две модели данных: реляционная модель в PostgreSQL и документная модель в MongoDB.

\subsection{Реляционная модель данных}

Структура реляционной базы данных предназначена для хранения основной метаинформации о сущностях системы и связей между ними. 
Основные таблицы:

\begin{itemize}
    \item \texttt{users}: информация о пользователях системы (id, name, surname, login, password(hash), photo).
    \item \texttt{projects}: описание проектов, создаваемых пользователями. Каждый проект связан с пользователем-владельцем (id, name, date, owner\_id).
    \item \texttt{file\_types}: типы создаваемых в системе файлов или документов (id, name, например, "заметка", "схема", "презентация").
    \item \texttt{files}: метаинформация о каждом файле или документе, независимо от его внутреннего содержания. Включает общие атрибуты (id, name, type\_id, author\_id, date) и ссылку \texttt{superObjectId} на соответствующий документ в MongoDB, где хранится сам контент.
    \item \texttt{projects\_users}: связующая таблица для реализации ролевой модели доступа пользователей к проектам, определяющая роль каждого участника в конкретном проекте (id, project\_id, user\_id, role).
    \item \texttt{projects\_files}: связующая таблица, указывающая на принадлежность файлов (из таблицы \texttt{files}) к проектам (из таблицы \texttt{projects}) (id, project\_id, file\_id).
\end{itemize}

Эта структура обеспечивает ссылочную целостность, транзакционность и возможности для сложных запросов с объединением данных. 
\clearpage
Полная реляционная схема базы данных представлена на рисунке~\ref{img:db-1}.

\includeimage
	{db-1}
	{f}
	{H}
	{1\textwidth}
	{Диаграмма реляционной базы данных}

\subsection{Документная модель данных (MongoDB)}

Документо-ориентированная модель данных в MongoDB используется для хранения непосредственно контента создаваемых сущностей, а также сложных, гибко структурированных данных, связанных с их представлением и стилизацией.
\clearpage
Основные коллекции:

\begin{enumerate}
    \item \textbf{\texttt{super\_objects}:} коллекция, содержащая основную информацию для каждого файла, метаданные которого хранятся в PostgreSQL.
    \begin{enumerate}
        \item \texttt{id}: уникальный идентификатор документа MongoDB.
        \item \texttt{fileId}: внешний ключ (индексированный и уникальный), связывающий данный документ с соответствующей записью в таблице \texttt{files} PostgreSQL.
        \item \texttt{name}: название документа, которое может синхронизироваться с \texttt{files.name} или быть специфичным.
        \item \texttt{serviceType}: ключевой атрибут, определяющий тип сервиса и структуру контента (например, \texttt{"note"}, \texttt{"scheme"}, \texttt{"presentation"}, \texttt{"mindmap"}, \texttt{"calendar"}, \texttt{"tracker"}, \texttt{"chat"}).
        \item \texttt{lastChangeDate}: время последнего изменения.
        \item \texttt{stylesMapId}: (опционально) ссылка на документ из коллекции \texttt{styles\_maps}, определяющий примененные к элементам контента стили.
        \item Для блочно-ориентированных сервисов (\texttt{serviceType = "note"}, \texttt{"presentation"}, \texttt{"chat"}):
        \begin{enumerate}
            \item \texttt{firstItem}: идентификатор первого блока контента (\texttt{ContentBlock}) в последовательности.
            \item \texttt{lastItem}: идентификатор последнего блока контента (\texttt{ContentBlock}).
        \end{enumerate}
        \item Для других сервисов поля \texttt{firstItem} и \texttt{lastItem} могут не использоваться, а контент будет структурирован в специфичных для сервиса коллекциях, связанных с \texttt{SuperObject} по его \texttt{id} или \texttt{fileId}.
    \end{enumerate}
    \item \textbf{\texttt{content\_blocks}:} коллекция, предназначенная для хранения отдельных блоков контента, используется для таких сервисов, как заметки, слайды презентаций, или сообщения в чате.
    \begin{enumerate}
        \item \texttt{id}: уникальный идентификатор блока.
        \item \texttt{objectType}: строковый идентификатор типа блока (например, \texttt{"paragraph"}, \texttt{"header"}, \texttt{"image"}, \texttt{"list"}, \texttt{"chatMessage"}).
        \item \texttt{data}: объект с парами ключ-значение со специфичными данными для данного \texttt{objectType}.
        \item \texttt{nextItem}, \texttt{prevItem}: идентификаторы для связывания блоков в двунаправленный список, формирующий последовательный контент.
    \end{enumerate}

\item Специфичные коллекции для неблочных сервисов: для поддержки функциональности, выходящей за рамки простой блочной структуры, вводятся дополнительные специализированные коллекции:
    \begin{enumerate}
        \item Для схем и mindmap (\texttt{serviceType = "scheme"}, \texttt{"mindmap"}):
            \begin{enumerate}
                \item \texttt{nodes}: коллекция узлов (фигур) с их атрибутами (тип, координаты, размеры, содержимое). Каждый узел связан с родительским \texttt{SuperObject}.
                \item \texttt{edges}: коллекция ребер (связей) между узлами, также с атрибутами (тип, узлы-источник и цель, метки) и связью с \texttt{SuperObject}.
            \end{enumerate}
        \item Для календаря (\texttt{serviceType = "calendar"}):
            \begin{enumerate}
                \item \texttt{calendar\_events}: коллекция событий с атрибутами (название, время начала/окончания, описание, местоположение, участники, правила повторения). Каждое событие связано с \texttt{SuperObject}, представляющим календарь.
            \end{enumerate}
        \item Для трекера задач (\texttt{serviceType = "tracker"}):
            \begin{enumerate}
                \item \texttt{task\_columns}: коллекция колонок (статусов) на доске задач, каждая связана с \texttt{SuperObject} доски.
                \item \texttt{task\_items}: коллекция задач с их атрибутами (название, описание, исполнители, сроки, приоритет, вложенные файлы, подзадачи), каждая задача принадлежит определенной колонке и связана с \texttt{SuperObject} доски.
             \end{enumerate}
    \end{enumerate}
\item Система стилизации: для обеспечения гибкой настройки внешнего вида различных элементов контента предлагается следующая структура коллекций:
    \begin{enumerate}
        \item \textbf{\texttt{styles}:} коллекция с определениями конкретных стилей. Каждый документ стиля включает:
            \begin{enumerate}
                \item \texttt{id}: уникальный идентификатор стиля.
                \item \texttt{targetType}: тип целевого объекта, к которому применим стиль (например, \texttt{"text\_block"}, \texttt{"scheme\_node"}, \texttt{"calendar\_event"}), что позволяет группировать релевантные атрибуты.
                \item \texttt{attributes}: объект с парами ключ-значение, описывающими визуальные свойства (например, \texttt{color}, \texttt{fontSize}, \texttt{backgroundColor}, \texttt{borderStyle}, специфичные для \texttt{targetType} атрибуты).
            \end{enumerate}
        \item \textbf{\texttt{styles\_maps}:} коллекция, обеспечивающая связь между элементами контента и определенными стилями. Каждый документ (один на \texttt{SuperObject}, связанный через \texttt{stylesMapId}) содержит:
            \begin{enumerate}
                \item \texttt{id}: уникальный идентификатор карты стилей.
                \item \texttt{links}: массив объектов \texttt{style\_link}.
            \end{enumerate}
        \item \textbf{\texttt{style\_link}}: описывает применение конкретного стиля к элементу:
            \begin{enumerate}
                \item \texttt{elementId}: идентификатор элемента (например, \texttt{ContentBlock.id}, \texttt{Node.id}), к которому применяется стиль.
                \item \texttt{styleId}: ссылка на идентификатор документа в коллекции \texttt{styles}.
                \item \texttt{scope}: (опционально) уточнение части элемента, к которой применяется стиль (например, \texttt{"background"}, \texttt{"text"}, \texttt{"border"}).
                \item \texttt{state}: (опционально) Для описания стилей интерактивных состояний (например, \texttt{"hover"}, \texttt{"active"}).
            \end{enumerate}
    \end{enumerate}
\end{enumerate}

Эта модель обеспечивает гибкость для хранения сложного и разнообразного контента. 
\clearpage
Схема коллекций этой базы данных представлена на рисунке~\ref{img:db-2}.

\includeimage
	{db-2}
	{f}
	{H}
	{1.2\textwidth, angle=90}
	{Схема коллекций нереляционной базы данных}

\section*{Вывод}

В рамках конструкторского раздела были разработаны и описаны основные архитектурные решения для программно-алгоритмического комплекса совместной работы и управления проектами.

Определены ключевые положения, включая выбор клиент-серверной архитектуры, гибридной модели хранения данных с использованием PostgreSQL для структурированной информации и MongoDB для гибкого контента, а также комбинированного подхода к взаимодействию через REST API и WebSocket для поддержки реального времени. 
Представлена общая структура приложения, выделяющая клиентскую и серверную части, а также многоуровневую организацию серверного приложения (контроллеры, сервисы, доступ к данным), что способствует модульности и разделению ответственности.

Описаны основные компоненты серверного приложения (сервисы управления пользователями, проектами, файлами, контентом и др.) и принципы их взаимодействия, проиллюстрированные на примере сценария совместного редактирования с помощью диаграммы последовательности. 
Зафиксированы ключевые структуры данных для реляционной (PostgreSQL) и документной (MongoDB) моделей, обеспечивающие хранение всей необходимой информации – от метаданных до сложного содержимого файлов и стилей. 
