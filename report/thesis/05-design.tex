\chapter{Конструкторский раздел}

\section{Основные положения предлагаемого программно-алгоритмического комплекса}

При проектировании программно-алгоритмического комплекса были заложены следующие основные положения, направленные на обеспечение его функциональности, масштабируемости и многоцелевого использования:

\begin{enumerate}[wide=12.5mm, leftmargin=12.5mm]
    \item \textbf{Клиент-серверная архитектура:} Комплекс реализуется в рамках клиент-серверной архитектуры, где пользователь взаимодействует с системой через клиентское приложение (веб-браузер), а основная бизнес-логика, управление данными и синхронизация выполняются на серверной стороне.
    \item \textbf{Гибридная модель хранения данных:} Учитывая необходимость работы с неоднородными данными, применяется гибридный подход к хранению:
        \begin{itemize}
            \item \textbf{PostgreSQL} используется для хранения структурированных, реляционных данных, требующих строгой схемы, целостности и поддержки транзакций. Сюда относятся данные о пользователях, проектах, файлах (метаданные), типах файлов, а также связи между ними (участие пользователей в проектах, принадлежность файлов проектам, роли и права доступа).
            \item \textbf{MongoDB} используется для хранения полуструктурированных и неструктурированных данных, требующих гибкости схемы и горизонтальной масштабируемости. В MongoDB хранится фактическое содержимое файлов, состоящее из различных блоков данных, а также информация о стилях.
        \end{itemize}
    \item \textbf{Комбинированное использование REST API и WebSockets:} Для взаимодействия между клиентом и сервером применяются два основных механизма:
        \begin{itemize}
            \item \textbf{REST API} используется для выполнения стандартных CRUD-операций (создание, чтение, обновление, удаление) над основными сущностями системы (пользователи, проекты, файлы и т.д.), а также для запроса или модификации данных, не требующих немедленной синхронизации у других пользователей.
            \item \textbf{WebSocket} используется для обеспечения взаимодействия в реальном времени важного для функций совместной работы: одновременное редактирование документов, обмен сообщениями в чате, доставка мгновенных уведомлений, отображение статуса присутствия пользователей.
        \end{itemize}
    \item \textbf{Модульность серверного приложения:} Серверное приложение проектируется с учетом модульности, где каждая основная функциональная область (управление пользователями, проектами, контентом и т.д.) выделяется в логический компонент, что упрощает разработку, тестирование и дальнейшее развитие системы.
    \item \textbf{Масштабируемость:} Архитектура и выбор технологий (в частности, использование MongoDB и возможность репликации/масштабирования PostgreSQL) закладывают основу для потенциальной горизонтальной и вертикальной масштабируемости системы при росте нагрузки.
\end{enumerate}

\section{Структура и компоненты программного приложения}

Программно-алгоритмический комплекс строится на основе многоуровневой архитектуры, близкой к принципам чистой или луковой архитектуры, с четким разделением ответственности между слоями.
Общая структура представлена на рисунке~\ref{img:structure-common}.

\includeimage
	{structure-common}
	{f}
	{H}
	{1\textwidth}
	{Общая архитектура программно-алгоритмического комплекса}


Основные компоненты и слои архитектуры:

\begin{enumerate}[wide=12.5mm, leftmargin=12.5mm]
    \item \textbf{Клиентское приложение} будет реализовано как одностраничное веб-приложение (SPA), отвечающее за пользовательский интерфейс, визуализацию данных и взаимодействие с пользователем. Обмен данными с серверной частью будет осуществляется через REST API и WebSocket.
    \item \textbf{Серверное приложение} реализует бизнес-логику и будет состоять из следующих слоев:
        \begin{itemize}
            \item \textbf{Слой Контроллеров}, который отвечает за обработку входящих HTTP-запросов и вызов методов соответствующего сервиса. В этот же слой логически входит обработка WebSocket-соединений для real-time взаимодействия.
            \item \textbf{Слой Сервисов}, который инкапсулирует основную бизнес-логику приложения. Сервисы координируют работу репозиториев, выполняют преобразование данных, реализуют правила предметной области и управляют транзакциями.
            \item \textbf{Слой Доступа к Данным}, включающий интерфейсы репозиториев и абстрагирующий детали взаимодействия с базами данных, предоставляя методы для CRUD-операций.
        \end{itemize}
    \item \textbf{Система хранения реляционных данных} хранит метаданные и связи между основными сущностями системы.
    \item \textbf{Система хранения документных данных:} хранит гибкое содержимое файлов и связанные с ним стили.
\end{enumerate}

На рисунке~\ref{img:components} представлена более подробная диаграмма компонентов серверного приложения.

\includeimage
	{components}
	{f}
	{H}
	{1\textwidth}
	{Диаграмма компонентов серверного приложения}


\section{Сценарий совместного редактирования}

Для иллюстрации взаимодействия компонентов системы при выполнении операций в реальном времени рассмотрим сценарий совместного редактирования документа. 
Процесс представлен диаграммой последовательности на рисунке~\ref{img:sequence}.

\includeimage
	{sequence}
	{f}
	{H}
	{1\textwidth}
	{Диаграмма последовательности: Совместное редактирование документа}

Основные шаги взаимодействия при внесении изменения пользователем~\textit{1} в документ, который также просматривает пользователь~\textit{2}:

\begin{enumerate}
    \item Пользователь~\textit{1} выполняет действие редактирования (например, вводит символ) в клиентском приложении~\textit{1}.
    \item Клиентское приложение~\textit{1} формирует сообщение об операции редактирования, указывая тип операции, позицию, измененные данные и идентификатор документа/файла, и отправляет его на сервер через установленное WebSocket-соединение.
    \item Серверное приложение, в частности его компонент, отвечающий за обработку WebSocket-сообщений, принимает данное сообщение.
    \item Обработчик WebSocket вызывает соответствующий метод сервисного слоя, ответственного за управление контентом, передавая ему детали операции редактирования.
    \item Сервис контента проверяет полученную операцию и взаимодействует со слоем доступа к данным для сохранения изменений в соответствующей системе хранения.
    \item После подтверждения успешного сохранения данных от слоя доступа к данным, компонент обработки WebSocket-сообщений получает уведомление об успешном применении операции.
    \item Компонент обработки WebSocket-сообщений рассылает информацию о выполненной операции всем остальным пользователям, включая пользователя~\textit{2}, которые в данный момент работают с этим же документом и подключены к соответствующей WebSocket-сессии.
    \item Клиентское приложение~\textit{2} получает сообщение об операции через WebSocket и применяет соответствующие изменения к своему локальному представлению документа, отображая актуальное состояние пользователю~\textit{2}.
\end{enumerate}

Данный механизм обеспечивает синхронизацию состояния документа у всех активных пользователей в режиме реального времени.

\section{Ключевые структуры данных}

Для хранения информации выбраны две модели данных: реляционная модель в PostgreSQL и документная модель в MongoDB.

\subsection{Реляционная модель данных}

Структура реляционной базы данных предназначена для хранения основной метаинформации о сущностях системы и связей между ними. 
Основные таблицы:

\begin{itemize}
    \item \texttt{users}: Информация о пользователях (id, name, surname, login, password(hash), photo).
    \item \texttt{projects}: Информация о проектах (id, author\_id, date, name).
    \item \texttt{file\_types}: Справочник типов файлов (id, name).
    \item \texttt{files}: Метаинформация о файлах (id, name, type\_id, author\_id, date).
    \item \texttt{projects\_users}: Связь пользователей с проектами, включая роли и разрешения (id, project\_id, user\_id, role, permission).
    \item \texttt{projects\_files}: Связь файлов с проектами (id, project\_id, file\_id).
\end{itemize}

\clearpage
Эта структура обеспечивает ссылочную целостность, транзакционность и возможности для сложных запросов с объединением данных. 
Полная реляционная схема базы данных представлена на рисунке~\ref{img:db-1}.

\includeimage
	{db-1}
	{f}
	{H}
	{1\textwidth}
	{Диаграмма реляционной базы данных}

\subsection{Документная модель данных (MongoDB)}

Структура данных в нереляционной базе данных предназначена для хранения содержимого файлов и связанной с ним информации. 
Используются коллекции документов, структура которых может гибко изменяться.
\clearpage
Основные коллекции:

\begin{itemize}
    \item \texttt{super\_objects}: Корневой документ для содержимого файла. Содержит fileId (связь с реляционной базой), метаданные контента (serviceType, lastChangeDate, name), ссылки на первый и последний блоки (firstItem, lastItem), карту стилей (stylesMapId) и другую служебную информацию (template, decoration, checkSum).
    \item \texttt{content\_blocks}: Представляет собой отдельные блоки контента (текст, список, изображение и т.д.), связанные в последовательность через prevItem и nextItem. Хранит фактические данные блока (data, items, label) и его тип (objectType).
    \item \texttt{styles}: Определения стилей с различными атрибутами форматирования.
    \item \texttt{styles\_maps}: Карты стилей, содержащие массив ссылок links, указывающих, какой стиль (styleId) применяется к какому блоку (textId) и в каком диапазоне (start, end).
\end{itemize}

Эта модель обеспечивает гибкость для хранения сложного и разнообразного контента. 
\clearpage
Схема коллекций этой базы данных представлена на рисунке~\ref{img:db-2}.

\includeimage
	{db-2}
	{f}
	{H}
	{1\textwidth}
	{Схема коллекций нереляционной базы данных}

\section*{Вывод}

В рамках конструкторского раздела были разработаны и описаны основные архитектурные решения для программно-алгоритмического комплекса совместной работы и управления проектами.

Определены ключевые положения, включая выбор клиент-серверной архитектуры, гибридной модели хранения данных с использованием PostgreSQL для структурированной информации и MongoDB для гибкого контента, а также комбинированного подхода к взаимодействию через REST API и WebSocket для поддержки реального времени. 
Представлена общая структура приложения, выделяющая клиентскую и серверную части, а также многоуровневую организацию серверного приложения (контроллеры, сервисы, доступ к данным), что способствует модульности и разделению ответственности.

Описаны основные компоненты серверного приложения (сервисы управления пользователями, проектами, файлами, контентом и др.) и принципы их взаимодействия, проиллюстрированные на примере сценария совместного редактирования с помощью диаграммы последовательности. 
Зафиксированы ключевые структуры данных для реляционной (PostgreSQL) и документной (MongoDB) моделей, обеспечивающие хранение всей необходимой информации – от метаданных до сложного содержимого файлов и стилей. 
